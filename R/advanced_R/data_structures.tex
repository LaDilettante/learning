\documentclass{article}\usepackage[]{graphicx}\usepackage[]{color}
%% maxwidth is the original width if it is less than linewidth
%% otherwise use linewidth (to make sure the graphics do not exceed the margin)
\makeatletter
\def\maxwidth{ %
  \ifdim\Gin@nat@width>\linewidth
    \linewidth
  \else
    \Gin@nat@width
  \fi
}
\makeatother

\definecolor{fgcolor}{rgb}{0.345, 0.345, 0.345}
\newcommand{\hlnum}[1]{\textcolor[rgb]{0.686,0.059,0.569}{#1}}%
\newcommand{\hlstr}[1]{\textcolor[rgb]{0.192,0.494,0.8}{#1}}%
\newcommand{\hlcom}[1]{\textcolor[rgb]{0.678,0.584,0.686}{\textit{#1}}}%
\newcommand{\hlopt}[1]{\textcolor[rgb]{0,0,0}{#1}}%
\newcommand{\hlstd}[1]{\textcolor[rgb]{0.345,0.345,0.345}{#1}}%
\newcommand{\hlkwa}[1]{\textcolor[rgb]{0.161,0.373,0.58}{\textbf{#1}}}%
\newcommand{\hlkwb}[1]{\textcolor[rgb]{0.69,0.353,0.396}{#1}}%
\newcommand{\hlkwc}[1]{\textcolor[rgb]{0.333,0.667,0.333}{#1}}%
\newcommand{\hlkwd}[1]{\textcolor[rgb]{0.737,0.353,0.396}{\textbf{#1}}}%

\usepackage{framed}
\makeatletter
\newenvironment{kframe}{%
 \def\at@end@of@kframe{}%
 \ifinner\ifhmode%
  \def\at@end@of@kframe{\end{minipage}}%
  \begin{minipage}{\columnwidth}%
 \fi\fi%
 \def\FrameCommand##1{\hskip\@totalleftmargin \hskip-\fboxsep
 \colorbox{shadecolor}{##1}\hskip-\fboxsep
     % There is no \\@totalrightmargin, so:
     \hskip-\linewidth \hskip-\@totalleftmargin \hskip\columnwidth}%
 \MakeFramed {\advance\hsize-\width
   \@totalleftmargin\z@ \linewidth\hsize
   \@setminipage}}%
 {\par\unskip\endMakeFramed%
 \at@end@of@kframe}
\makeatother

\definecolor{shadecolor}{rgb}{.97, .97, .97}
\definecolor{messagecolor}{rgb}{0, 0, 0}
\definecolor{warningcolor}{rgb}{1, 0, 1}
\definecolor{errorcolor}{rgb}{1, 0, 0}
\newenvironment{knitrout}{}{} % an empty environment to be redefined in TeX

\usepackage{alltt}

\usepackage{verbatim}

\title{Advanced R - Data structure}
\date{\today}
\author{Anh Le}
\IfFileExists{upquote.sty}{\usepackage{upquote}}{}
\begin{document}
\maketitle


\section{Vector}

\verb`is.vector()` does not test if an object is a vector. Instead it returns \verb`TRUE` only if the object is a vector with no attributes apart from names.

\begin{knitrout}
\definecolor{shadecolor}{rgb}{0.969, 0.969, 0.969}\color{fgcolor}\begin{kframe}
\begin{alltt}
\hlstd{real_vector} \hlkwb{<-} \hlkwd{c}\hlstd{(}\hlnum{1}\hlstd{,} \hlnum{2}\hlstd{,} \hlnum{3}\hlstd{)}
\hlkwd{is.vector}\hlstd{(real_vector)}
\end{alltt}
\begin{verbatim}
## [1] TRUE
\end{verbatim}
\begin{alltt}
\hlkwd{attr}\hlstd{(real_vector,} \hlstr{"someattr"}\hlstd{)} \hlkwb{<-} \hlstr{"attrvalue"}
\hlkwd{is.vector}\hlstd{(real_vector)}
\end{alltt}
\begin{verbatim}
## [1] FALSE
\end{verbatim}
\begin{alltt}
\hlkwd{is.atomic}\hlstd{(real_vector)}
\end{alltt}
\begin{verbatim}
## [1] TRUE
\end{verbatim}
\end{kframe}
\end{knitrout}

\subsection{Atomic vector}

\verb`NA` is a logical vector of length one. \verb`NA` is coerced to the appropriate type to blend in with the rest of the vector.

\begin{knitrout}
\definecolor{shadecolor}{rgb}{0.969, 0.969, 0.969}\color{fgcolor}\begin{kframe}
\begin{alltt}
\hlkwd{typeof}\hlstd{(}\hlnum{NA}\hlstd{)}
\end{alltt}
\begin{verbatim}
## [1] "logical"
\end{verbatim}
\begin{alltt}
\hlstd{v} \hlkwb{<-} \hlkwd{c}\hlstd{(}\hlnum{1}\hlstd{,} \hlnum{2}\hlstd{,} \hlnum{NA}\hlstd{,} \hlnum{3}\hlstd{);} \hlkwd{typeof}\hlstd{(v[}\hlnum{3}\hlstd{])}
\end{alltt}
\begin{verbatim}
## [1] "double"
\end{verbatim}
\begin{alltt}
\hlstd{v} \hlkwb{<-} \hlkwd{c}\hlstd{(}\hlnum{1L}\hlstd{,} \hlnum{2L}\hlstd{,} \hlnum{NA}\hlstd{,} \hlnum{3L}\hlstd{);} \hlkwd{typeof}\hlstd{(v[}\hlnum{3}\hlstd{])}
\end{alltt}
\begin{verbatim}
## [1] "integer"
\end{verbatim}
\end{kframe}
\end{knitrout}

\section{List}

Lists are sometimes called \textbf{recursive} vectors, in contrast to regular \textbf{atomic} vectors.

\begin{knitrout}
\definecolor{shadecolor}{rgb}{0.969, 0.969, 0.969}\color{fgcolor}\begin{kframe}
\begin{alltt}
\hlstd{x} \hlkwb{<-} \hlkwd{list}\hlstd{(}\hlkwd{list}\hlstd{(}\hlkwd{list}\hlstd{()))}
\hlkwd{str}\hlstd{(x)}
\end{alltt}
\begin{verbatim}
## List of 1
##  $ :List of 1
##   ..$ : list()
\end{verbatim}
\begin{alltt}
\hlkwd{is.recursive}\hlstd{(x)}
\end{alltt}
\begin{verbatim}
## [1] TRUE
\end{verbatim}
\end{kframe}
\end{knitrout}

\verb`c()` combines several list into one. If we try to combine vector and list, the vector will be coerced to list.

\begin{knitrout}
\definecolor{shadecolor}{rgb}{0.969, 0.969, 0.969}\color{fgcolor}\begin{kframe}
\begin{alltt}
\hlstd{x} \hlkwb{<-} \hlkwd{list}\hlstd{(}\hlkwd{list}\hlstd{(}\hlnum{1}\hlstd{,} \hlnum{2}\hlstd{),} \hlkwd{c}\hlstd{(}\hlnum{3}\hlstd{,} \hlnum{4}\hlstd{))}
\hlstd{y} \hlkwb{<-} \hlkwd{c}\hlstd{(}\hlkwd{list}\hlstd{(}\hlnum{1}\hlstd{,} \hlnum{2}\hlstd{),} \hlkwd{c}\hlstd{(}\hlnum{3}\hlstd{,} \hlnum{4}\hlstd{))}
\hlkwd{str}\hlstd{(x)}
\end{alltt}
\begin{verbatim}
## List of 2
##  $ :List of 2
##   ..$ : num 1
##   ..$ : num 2
##  $ : num [1:2] 3 4
\end{verbatim}
\begin{alltt}
\hlkwd{str}\hlstd{(y)}
\end{alltt}
\begin{verbatim}
## List of 4
##  $ : num 1
##  $ : num 2
##  $ : num 3
##  $ : num 4
\end{verbatim}
\end{kframe}
\end{knitrout}

\section{Exercises}

\begin{enumerate}
\item What are the six types of atomic vector? How does a list differ from an atomic vector?

Six types: logical, double, integer, character, and complex, raw. List is recursive, can hold multiple typles

\item What makes \verb`is.vector()` and \verb`is.numeric()` fundamentally different to \verb`is.list()` and \verb`is.character()`?

\item Test knowledge of vector coercion rule

\begin{knitrout}
\definecolor{shadecolor}{rgb}{0.969, 0.969, 0.969}\color{fgcolor}\begin{kframe}
\begin{alltt}
\hlkwd{c}\hlstd{(}\hlnum{1}\hlstd{,} \hlnum{FALSE}\hlstd{)} \hlcom{# Should be c(1, 0)}
\end{alltt}
\begin{verbatim}
## [1] 1 0
\end{verbatim}
\begin{alltt}
\hlkwd{c}\hlstd{(}\hlstr{"a"}\hlstd{,} \hlnum{1}\hlstd{)} \hlcom{# Should be c("a", "1")}
\end{alltt}
\begin{verbatim}
## [1] "a" "1"
\end{verbatim}
\begin{alltt}
\hlkwd{c}\hlstd{(}\hlkwd{list}\hlstd{(}\hlnum{1}\hlstd{),} \hlstr{"a"}\hlstd{)} \hlcom{# Should be list(1, "a"), cuz "a" is coerced to list first}
\end{alltt}
\begin{verbatim}
## [[1]]
## [1] 1
## 
## [[2]]
## [1] "a"
\end{verbatim}
\begin{alltt}
\hlkwd{str}\hlstd{(}\hlkwd{c}\hlstd{(}\hlnum{TRUE}\hlstd{,} \hlnum{1L}\hlstd{))} \hlcom{# Should be c(1L, 1L)}
\end{alltt}
\begin{verbatim}
##  int [1:2] 1 1
\end{verbatim}
\end{kframe}
\end{knitrout}

\item Why do you need to use \verb`unlist()` to convert a list to an atomic vector? Why doesn't \verb`as.vector()` work?

Probably because \verb`list` is already a (recursive, non-atomic) vector. Indeed,

\begin{knitrout}
\definecolor{shadecolor}{rgb}{0.969, 0.969, 0.969}\color{fgcolor}\begin{kframe}
\begin{alltt}
\hlstd{l} \hlkwb{<-} \hlkwd{list}\hlstd{(}\hlnum{1}\hlstd{,} \hlnum{2}\hlstd{)}
\hlkwd{is.vector}\hlstd{(l)}
\end{alltt}
\begin{verbatim}
## [1] TRUE
\end{verbatim}
\begin{alltt}
\hlkwd{is.vector}\hlstd{(l,} \hlkwc{mode}\hlstd{=}\hlstr{"logical"}\hlstd{)}
\end{alltt}
\begin{verbatim}
## [1] FALSE
\end{verbatim}
\begin{alltt}
\hlkwd{is.vector}\hlstd{(l,} \hlkwc{mode}\hlstd{=}\hlstr{"list"}\hlstd{)}
\end{alltt}
\begin{verbatim}
## [1] TRUE
\end{verbatim}
\begin{alltt}
\hlkwd{is.vector}\hlstd{(l,} \hlkwc{mode}\hlstd{=}\hlstr{"expression"}\hlstd{)}
\end{alltt}
\begin{verbatim}
## [1] FALSE
\end{verbatim}
\end{kframe}
\end{knitrout}

\item Why is \verb`1 == "1"` true? (Because \verb`1 (double)` is coerced to \verb`"1 (character)"`). Why is \verb`-1 < FALSE` true? (Because \verb`FALSE (logical)` is coerced to \verb`0 (double)`). Why is \verb`"one" < 2` false? (Because \verb`2` is coerced to \verb`"2"`, and strings are compared alphabetically)

\item Why is the default missing value, \verb`NA`, a logical vector? What's special about logical vectors?

Probably because it's the most flexible, so whenever it's put together with other values in the vectors, \verb`NA` will be coerced, not the other values.

\end{enumerate}

\section{Attributes}

\section{Factors}

Factors are built on top of integer vectors using two attributes: the \verb`class()`, "factor", which makes them behave differently from regular integer vectors, and the \verb`levels()`, which defines the set of allowed values.

\section{Exercises}

\begin{enumerate}
\item An early draft used this code to illustrate \verb`structure()`:

\begin{knitrout}
\definecolor{shadecolor}{rgb}{0.969, 0.969, 0.969}\color{fgcolor}\begin{kframe}
\begin{alltt}
\hlkwd{structure}\hlstd{(}\hlnum{1}\hlopt{:}\hlnum{5}\hlstd{,} \hlkwc{comment} \hlstd{=} \hlstr{"my attribute"}\hlstd{)}
\end{alltt}
\begin{verbatim}
## [1] 1 2 3 4 5
\end{verbatim}
\end{kframe}
\end{knitrout}

But when you print that object you don't see the comment attribute. Why?
\begin{knitrout}
\definecolor{shadecolor}{rgb}{0.969, 0.969, 0.969}\color{fgcolor}\begin{kframe}
\begin{alltt}
\hlstd{x} \hlkwb{<-} \hlkwd{structure}\hlstd{(}\hlnum{1}\hlopt{:}\hlnum{5}\hlstd{,} \hlkwc{comment}\hlstd{=}\hlstr{"my attribute"}\hlstd{)}
\hlkwd{comment}\hlstd{(x)}
\end{alltt}
\begin{verbatim}
## [1] "my attribute"
\end{verbatim}
\end{kframe}
\end{knitrout}

Turns out \verb`comment` is a special attribute that does not get printed. See \verb`help(comment)`.

\item What happens to a factor when you modify its levels?

\begin{knitrout}
\definecolor{shadecolor}{rgb}{0.969, 0.969, 0.969}\color{fgcolor}\begin{kframe}
\begin{alltt}
\hlstd{f1} \hlkwb{<-} \hlkwd{factor}\hlstd{(letters)}
\hlkwd{levels}\hlstd{(f1)} \hlkwb{<-} \hlkwd{rev}\hlstd{(}\hlkwd{levels}\hlstd{(f1))}
\hlstd{f1}
\end{alltt}
\begin{verbatim}
##  [1] z y x w v u t s r q p o n m l k j i h g f e d c b a
## Levels: z y x w v u t s r q p o n m l k j i h g f e d c b a
\end{verbatim}
\end{kframe}
\end{knitrout}

Both the observations and the level labels are switched to the new levels.

\item What does this code do? How do \verb`f2` and \verb`f3` differ from \verb`f1`

\begin{knitrout}
\definecolor{shadecolor}{rgb}{0.969, 0.969, 0.969}\color{fgcolor}\begin{kframe}
\begin{alltt}
\hlstd{f2} \hlkwb{<-} \hlkwd{rev}\hlstd{(}\hlkwd{factor}\hlstd{(letters))}
\hlstd{f3} \hlkwb{<-} \hlkwd{factor}\hlstd{(letters,} \hlkwc{levels}\hlstd{=}\hlkwd{rev}\hlstd{(letters))}
\hlstd{f2}
\end{alltt}
\begin{verbatim}
##  [1] z y x w v u t s r q p o n m l k j i h g f e d c b a
## Levels: a b c d e f g h i j k l m n o p q r s t u v w x y z
\end{verbatim}
\begin{alltt}
\hlstd{f3}
\end{alltt}
\begin{verbatim}
##  [1] a b c d e f g h i j k l m n o p q r s t u v w x y z
## Levels: z y x w v u t s r q p o n m l k j i h g f e d c b a
\end{verbatim}
\end{kframe}
\end{knitrout}

\verb`f2` only switches the order of the observations, the levels is the same as \verb`f1`. \verb`f3` has the same observations as \verb`f1`, but the levels is reversed.
\end{enumerate}

\section{Matrices and arrays}

\section{Exercises}

\begin{enumerate}
\item What does \verb`dim()` return when applied to a vector?

It returns \verb`NULL`

\item If \verb`is.matrix(x)` is \verb`TRUE`, what will `is.array(x)` return?

Must also be \verb`TRUE`.

\item How would you describe the following three objects? What makes them different to \verb`1:5`?

\begin{knitrout}
\definecolor{shadecolor}{rgb}{0.969, 0.969, 0.969}\color{fgcolor}\begin{kframe}
\begin{alltt}
\hlstd{x1} \hlkwb{<-} \hlkwd{array}\hlstd{(}\hlnum{1}\hlopt{:}\hlnum{5}\hlstd{,} \hlkwd{c}\hlstd{(}\hlnum{1}\hlstd{,} \hlnum{1}\hlstd{,} \hlnum{5}\hlstd{))}
\hlstd{x2} \hlkwb{<-} \hlkwd{array}\hlstd{(}\hlnum{1}\hlopt{:}\hlnum{5}\hlstd{,} \hlkwd{c}\hlstd{(}\hlnum{1}\hlstd{,} \hlnum{5}\hlstd{,} \hlnum{1}\hlstd{))}
\hlstd{x3} \hlkwb{<-} \hlkwd{array}\hlstd{(}\hlnum{1}\hlopt{:}\hlnum{5}\hlstd{,} \hlkwd{c}\hlstd{(}\hlnum{5}\hlstd{,} \hlnum{1}\hlstd{,} \hlnum{1}\hlstd{))}
\hlstd{x1} \hlcom{# 5 1x1 matrices}
\end{alltt}
\begin{verbatim}
## , , 1
## 
##      [,1]
## [1,]    1
## 
## , , 2
## 
##      [,1]
## [1,]    2
## 
## , , 3
## 
##      [,1]
## [1,]    3
## 
## , , 4
## 
##      [,1]
## [1,]    4
## 
## , , 5
## 
##      [,1]
## [1,]    5
\end{verbatim}
\begin{alltt}
\hlstd{x2} \hlcom{# 1 1x5 matrix}
\end{alltt}
\begin{verbatim}
## , , 1
## 
##      [,1] [,2] [,3] [,4] [,5]
## [1,]    1    2    3    4    5
\end{verbatim}
\begin{alltt}
\hlstd{x3} \hlcom{# 1 5x1 matrix}
\end{alltt}
\begin{verbatim}
## , , 1
## 
##      [,1]
## [1,]    1
## [2,]    2
## [3,]    3
## [4,]    4
## [5,]    5
\end{verbatim}
\end{kframe}
\end{knitrout}
\end{enumerate}

\section{Data frames}

Under the hood, a data frame is a list of equal-length vectors.

A data frame share properties with both the matrix and the list. For example, \verb`length()` of a data frame is the length of the underlying list, i.e equivalent to \verb`ncol()`. \verb`names()` is equivalent to \verb`colnames()`.

Same idea with subsetting -- we can subset a dataframe both in list-way (\verb`df$col`) and matrix-way (\verb`df[x, y]`).

Data frame is S3 class, thus its type reflects the underlying vector to build it, which is a list. Use \verb`class()` and \verb`is.data.frame()` to test.


\begin{knitrout}
\definecolor{shadecolor}{rgb}{0.969, 0.969, 0.969}\color{fgcolor}\begin{kframe}
\begin{alltt}
\hlstd{df} \hlkwb{<-} \hlkwd{data.frame}\hlstd{(}\hlkwc{x}\hlstd{=}\hlkwd{c}\hlstd{(}\hlnum{1}\hlstd{,}\hlnum{2}\hlstd{),} \hlkwc{y}\hlstd{=}\hlkwd{c}\hlstd{(}\hlnum{3}\hlstd{,}\hlnum{4}\hlstd{))}
\hlkwd{typeof}\hlstd{(df)}
\end{alltt}
\begin{verbatim}
## [1] "list"
\end{verbatim}
\begin{alltt}
\hlkwd{class}\hlstd{(df)}
\end{alltt}
\begin{verbatim}
## [1] "data.frame"
\end{verbatim}
\begin{alltt}
\hlkwd{is.data.frame}\hlstd{(df)}
\end{alltt}
\begin{verbatim}
## [1] TRUE
\end{verbatim}
\end{kframe}
\end{knitrout}

\section{Exercises}

\begin{enumerate}
\item What attributes does a data frame possess?

\begin{knitrout}
\definecolor{shadecolor}{rgb}{0.969, 0.969, 0.969}\color{fgcolor}\begin{kframe}
\begin{alltt}
\hlstd{df} \hlkwb{<-} \hlkwd{data.frame}\hlstd{(}\hlkwc{x} \hlstd{=} \hlkwd{c}\hlstd{(}\hlnum{1}\hlstd{,}\hlnum{2}\hlstd{))}
\hlkwd{attributes}\hlstd{(df)}
\end{alltt}
\begin{verbatim}
## $names
## [1] "x"
## 
## $row.names
## [1] 1 2
## 
## $class
## [1] "data.frame"
\end{verbatim}
\end{kframe}
\end{knitrout}

\item What does \verb`as.matrix()` do when applied to a data frame with columns of different types?

Probably coerced to the least flexible?

\begin{knitrout}
\definecolor{shadecolor}{rgb}{0.969, 0.969, 0.969}\color{fgcolor}\begin{kframe}
\begin{alltt}
\hlstd{df} \hlkwb{<-} \hlkwd{data.frame}\hlstd{(}\hlkwc{x} \hlstd{=} \hlkwd{c}\hlstd{(}\hlnum{1}\hlstd{,}\hlnum{2}\hlstd{),} \hlkwc{y}\hlstd{=}\hlkwd{c}\hlstd{(}\hlstr{"a"}\hlstd{,} \hlstr{"b"}\hlstd{))}
\hlkwd{as.matrix}\hlstd{(df)}
\end{alltt}
\begin{verbatim}
##      x   y  
## [1,] "1" "a"
## [2,] "2" "b"
\end{verbatim}
\end{kframe}
\end{knitrout}


\item Can you have a data frame with 0 rows? What about 0 columns?

\begin{knitrout}
\definecolor{shadecolor}{rgb}{0.969, 0.969, 0.969}\color{fgcolor}\begin{kframe}
\begin{alltt}
\hlstd{df_norows} \hlkwb{<-} \hlkwd{data.frame}\hlstd{(}\hlkwc{x}\hlstd{=}\hlkwd{numeric}\hlstd{(}\hlnum{0}\hlstd{),} \hlkwc{y}\hlstd{=}\hlkwd{numeric}\hlstd{(}\hlnum{0}\hlstd{))}
\hlstd{df_norows}
\end{alltt}
\begin{verbatim}
## [1] x y
## <0 rows> (or 0-length row.names)
\end{verbatim}
\begin{alltt}
\hlcom{# Cannot have no columns by itsef?}
\hlcom{# data.frame() returns a df with 0 col and 0 row}
\end{alltt}
\end{kframe}
\end{knitrout}

\end{enumerate}

\section{Subsetting and assignment}

You can't combine integer indices with NA but you can combine logical indices with NA (where it is treated as FALSE)

\begin{knitrout}
\definecolor{shadecolor}{rgb}{0.969, 0.969, 0.969}\color{fgcolor}\begin{kframe}
\begin{alltt}
\hlstd{x} \hlkwb{<-} \hlkwd{c}\hlstd{(}\hlnum{1}\hlstd{,} \hlnum{2}\hlstd{,} \hlnum{3}\hlstd{,} \hlnum{4}\hlstd{)}
\hlstd{x[}\hlkwd{c}\hlstd{(}\hlnum{1}\hlstd{,} \hlnum{NA}\hlstd{)]} \hlkwb{<-} \hlnum{10}
\hlkwd{x}\hlstd{(}\hlkwd{c}\hlstd{(T, F,} \hlnum{NA}\hlstd{,} \hlnum{NA}\hlstd{, T))} \hlkwb{<-} \hlnum{10}
\end{alltt}


{\ttfamily\noindent\bfseries\color{errorcolor}{\#\# Error: could not find function "{}x<-"{}}}\begin{alltt}
\hlstd{x}
\end{alltt}
\begin{verbatim}
## [1] 10  2  3  4
\end{verbatim}
\end{kframe}
\end{knitrout}



\end{document}
